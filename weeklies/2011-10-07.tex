\documentclass{article}

\usepackage{style}
\usepackage{url}

\title{Weeklie}
\author{Hallgeir Lien}

\begin{document}

\maketitle

After I sent my last weeklie, I spent a few hours on making a USGS DEM (Digital Elevation Map) format to raw converter, in order to use real world maps from Kartverket or the US. Related to this, I also made a normal map generator that takes a RAW file and outputs an RGB normal map.

Of the things that has taken a lot of time this week was debugging the snow simulation code from earlier master projects. I have found various bugs. First of all, in the snow particle update kernel, the accelleration in the X direction was used to compute the new position for all directions, even Y and Z. Second, there was a bug with the function that calls the random number generator (rand()) that caused it to return a NEGATIVE number between -1 and 0, instead of between 0 and 1. In addition to this, I have fixed some ugly cases where there might be a division by 0.

In addition to this, I have implemented a triangle mesh / model loader. This is for loading road models (and other models, like landmarks) for the snow simulator. This part is not completed yet; the mesh is loaded, but is not yet textured (but this will come shortly).

Next, I wrote an annotated bibliography for the paper Procedural Generation of Roads by E. Galin et. al. This is included in this document. 

Lastly, I created a presentation for Friday (TDT24), which I spent most of Thursday and Friday before 13:15. My presentation was postponed for Monday 17th, however.

Next week I plan to first of all fix the terrain so that the heights in the terrain is correct proportional to the dimensions. Currently this is not the case at all. This will allow roads to be imported more seamlessly. I will also try to get textures working for meshes.

Second of all, I will try to improve on the road generator I have made. Currently, it outputs a roadXML with slightly offset height values along the curve, because I need to compute the clothoid trajectory to know the height along the points on the curve, not only the control points.

\appendix
\section{Annotated bibliography of E. Galin et. al., Procedural Generation of Roads}

The authors of this paper presents an algorithm for generating roads using an A* search. They claim to be able to create realistic roads using a carefully designed cost function that takes into account slopes, curvature of the road trajectory, water depth and so on. The algorithm produces a path that minimizes this cost when integrated over the trajectory. 

The terrain is first discretized into a grid of $m\times n$ grid points, where each grid point represents a height in the terrain. Then, an A* search is performed on this terrain, where the nodes in the graph is each point in the discrete grid. In order to overcome the limit-on-direction problem, more neighbors for each gridpoint than just the 8 adjecent points are considered. After the discrete path is found, the continuous road trajectory is generated by using the points as control points in a clothoid spline. Finally, the terrain is excavated and the road is created using a procedurally generated model. 

The algorithm presented in this paper is the one implemented for this project.


%The problem is first reduced to a discrete search problem by discretizing the continous domain, i.e. the terrain, into $m\times n$ grid points. Furthermore, the neighborhood of each grid point ${\mathbf p}_{i,j}$ is defined as all grid points ${\mathbf p}_{i^\prime,j^\prime}$ where $|i-i^\prime|$, $M_k(i,j)$ where $ = \{{\mathbf p}_{i^\prime,j^\prime} : |i^\prime-i| \leq k \wedge |j^\prime-j| \leq k \wedge gcd(i^\prime,j^\prime) = 1, i^\prime,j^\prime \in \mathbb{Z}\}$


%The problem that is solved is the anisotropic shortest path problem, that is, 
%
%Anisotropic shortest path: Find the path P that minimizes the line integral of the cost function, i.e. int\_P c(x,x',x''). Optimization: Use paths formed by concatenated line segments. Use a k-neighborhood connectivity mask (i.e. use more than the nearest neighbors) in the computation of the shortest path. M(p) subset of {q, where ||p-q|| <= r}.
%
%A*:
%1. When priority queue Q is not empty, select the point p with the smallest cost value.
%2. If p == final destination, we're done.
%3. For all points q in M(p), evaluate c(p,q) (that is, the cost to go to p, then to q). If c(p,q) < c(q), then we have found a shorter path to q, and we set the predecessor of q to p.
%
%In step 3: Compute M(p). Also: Compute line integral from p to q. Discretizize into n intervals then integrate.
%
%Cost function: c(p,p',p''). We have functions k\_i(p,p',p'') that evaluate different characterisitics of the terrain, and output a number. E.g. slopes and curvatures. u\_i weights this number. Then we sum over all i to find the cost.
%
%Path segment masks: Choose path segments with length <= k such that gcd(i,j) = 1. 


\end{document}
