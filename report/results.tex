\chapter{Results}
\label{chap:results}
In this chapter I present visual results for the roads that are imported into the snow simulator, and some performance analysis of the road generator. 

\section{Test bench hardware}
All tests and renderings were performned on a desktop computer with the specifications given in table \ref{tab:testspecs}. 
\begin{table}[ht]
\centering
\begin{tabular}{ccc}
\hline
\tbf {Component} & \tbf {Specifications} & \tbf {Remarks}\\
\hline
CPU      & Intel Core(tm)2 Quad Q9550 2.83GHz & \\
L1 Cache & 64KB                               & 32KB data, 32KB instruction\\
L2 Cache & 12MB                               & Unified; two cores share 6MB\\
Memory   & 4GB & \\
GPU 1    & NVIDIA Quadro FX 5800              & \\
GPU 2    & NVIDIA Tesla C1060                 & \\
\hline
\end{tabular}
\caption{Hardware specifications for test bench}
\label{tab:testspecs}
\end{table}
The Tesla C1060 device was used for all GPU computing tasks.

\section{Performance of road generator}
In this section, I present results from performance testing of the road generator. 

\subsection{Testing parameters}
Four different maps have been tested; their properties are shown in table \ref{tab:testmaps}. These maps range from very small to very large, with different resolutions. As for weights 

\begin{table}[ht]
\centering
\begin{tabular}{lll}
\hline
\tbf{Map} & \tbf{Dimensions} & \tbf{Resolution (m)}\\
\hline
Mt. St. Helens before erruption & $256\times 256$ & $30$\\ 
Mt. St. Helens after erruption  & $768\times 768$ & $10$\\
Random fractal terrain & $1024\times 1024$ & $10$\\
Trondheim & $4096\times 4096$ & $20$\\
\hline
\end{tabular}
\caption{Height maps used for performance testing}
\label{tab:testmaps}
\end{table}

\begin{table}[ht]
\centering

\subfloat{
    \begin{tabular}{ccccc}
    \hline
    \tbf{Height per} & \tbf{Runningtime} & \tbf{Speedup} & \tbf{Road cost} & \tbf{Difference}\\
    \tbf{grid point} &                   &               &                 & \tbf{from "optimal"}\\
    \hline
    \input{data/helens_before}
    \hline
    \end{tabular}
}

\subfloat{
    \begin{tabular}{ccccc}
    \hline
    \tbf{Height per} & \tbf{Runningtime} & \tbf{Speedup} & \tbf{Road cost} & \tbf{Difference}\\
    \tbf{grid point} &                   &               &                 & \tbf{from "optimal"}\\
    \hline
    \input{data/helens_after}
    \hline
    \end{tabular}
}

\subfloat{
    \begin{tabular}{ccccc}
    \hline
    \tbf{Height per} & \tbf{Runningtime} & \tbf{Speedup} & \tbf{Road cost} & \tbf{Difference}\\
    \tbf{grid point} &                   &               &                 & \tbf{from "optimal"}\\
    \hline
    \input{data/mountains}
    \hline
    \end{tabular}
}

\subfloat{
    \begin{tabular}{ccccc}
    \hline
    \tbf{Height per} & \tbf{Runningtime} & \tbf{Speedup} & \tbf{Road cost} & \tbf{Difference}\\
    \tbf{grid point} &                   &               &                 & \tbf{from "optimal"}\\
    \hline
    \input{data/trondheim}
    \hline
    \end{tabular}
}

\caption{Effect of density of grid}
\label{tab:effect_of_density}
\end{table}


\section{Visual results}
In this section, we will look at visual results for the generated roads; both as an overlay to the heightmap, and some of the generated roads imported into the snow simulator. 

First, we look at the road trajectories that are generated for different height maps. These are shown in figure \ref{fig:road_trajectory}. Figure \ref{fig:trajectory_mountains} shows an automatically generated terrain (a fractal terrain) representing a mountainous terrain, and a valley. Figures \ref{fig:trajectory_helens_before} and \ref{fig:trajectory_helens_after} is heightmaps representing Mt. St. Helens before and after the erruption; this is terrains converted from actual digital elevation maps (see \cite{helens_dem} for the source DEMs).

\begin{figure}[ht]
\centering
\subfloat[Autogenerated mountainous terrain]{\label{fig:trajectory_mountains}\includegraphics[width=1\textwidth]{figure/generated_road_trajectory}}\\
\subfloat[Mount St. Helens (before erruption)] {\label{fig:trajectory_helens_before}\includegraphics[width=0.485\textwidth]{figure/generated_road_trajectory_2}}\quad
\subfloat[Mount St. Helens (after erruption)]{\label{fig:trajectory_helens_after}\includegraphics[width=0.485\textwidth]{figure/generated_road_trajectory_3}}
\caption{Procedurally generated road trajectories for different heightmaps}
\label{fig:road_trajectory}
\end{figure}

After loading the model into the snow simulator and adjusting the terrain, we can see the roads as it curves through the terrain. Figure \ref{fig:road_in_terrain_nosnow} shows the road and the terrain without snow cover. We clearly see the effects of the terrain adjustments in the bottom two screenshots, i.e. figures \ref{fig:road_in_terrain_nosnow_2} and \ref{fig:road_in_terrain_nosnow_3}. Note that snow rendering was disabled on the overview image (figure \ref{fig:road_in_terrain_nosnow_1}) in order to improve visibility of the road.

\begin{figure}[ht]
\centering
\subfloat[Terrain overview]{\label{fig:road_in_terrain_nosnow_1}\includegraphics[width=\textwidth]{figure/screenshots/road_nosnow_1}}\\
\subfloat[Detailed view of road]{\includegraphics[width=0.485\textwidth]{figure/screenshots/road_nosnow_2}}\quad
\subfloat[Significant terrain adjustments]{\includegraphics[width=0.485\textwidth]{figure/screenshots/road_nosnow_3}}
\caption{Road and terrain before snowcover}
\label{fig:road_in_terrain_nosnow}
\end{figure}

After some snow has fallen, as in figure \ref{fig:road_in_terrain_nosnow}, the terrain has been more or less covered in snow except for some shadowed areas. The snow height is continously "smoothed", so that snow that falls in areas with a steep gradient in the terrain, is moved downwards along the gradient, simulating sliding snow (think of it as a mini avalanche). We clearly see this effect on the road in figures \ref{fig:road_in_terrain_nosnow_2} and \ref{fig:road_in_terrain_nosnow_3}.

\begin{figure}[ht]
\centering
\subfloat[Snow covered terrain overview]{\label{fig:road_in_terrain_nosnow_1}\includegraphics[width=\textwidth]{figure/screenshots/road_snow_1}}\\
\subfloat[Snow from higher up covering road]{\label{fig:road_in_terrain_nosnow_2}\includegraphics[width=0.485\textwidth]{figure/screenshots/road_snow_2}}\quad
\subfloat[Another case of snow sliding down on the road]{\label{fig:road_in_terrain_nosnow_3}\includegraphics[width=0.485\textwidth]{figure/screenshots/road_snow_3}}
\caption{Road and terrain with snowcover}
\label{fig:road_in_terrain_snow}
\end{figure}
