\chapter{Results}
\label{chap:results}
In this chapter I present visual results for the roads that are imported into the snow simulator, and some performance analysis of the road generator. 

\section{Test bench hardware}
All tests and renderings were performned on a desktop computer with the specifications given in table \ref{tab:testspecs}. 
\begin{table}[ht]
\centering
\begin{tabular}{lll}
\hline
\tbf {Component} & \tbf {Specifications} & \tbf {Remarks}\\
\hline
CPU      & Intel Core(tm)2 Quad Q9550 2.83GHz & \\
L1 Cache & 64KB                               & 32KB data, 32KB instruction\\
L2 Cache & 12MB                               & Unified; two cores share 6MB\\
Memory   & 4GB & \\
GPU 1    & NVIDIA Quadro FX 5800              & \\
GPU 2    & NVIDIA Tesla C1060                 & GPU used for the GPU computing tasks\\
\hline
\end{tabular}
\caption{Hardware specifications for test bench}
\label{tab:testspecs}
\end{table}
The Tesla C1060 device was used for all GPU computing tasks. The GeForce Quadro FX 5800 was used for rendering.

\section{Visual results}
In this section, we will look at visual results for the generated roads; both as an overlay to the heightmap, and some of the generated roads imported into the snow simulator. %All roads have a starting point at position $(0,0)$, i.e. lower left corner of the heightmap, and a end position at the upper right corner.

First, we look at the road trajectories that are generated for different height maps. These are shown in figure \ref{fig:road_trajectory}. Figure \ref{fig:trajectory_mountains} shows an automatically generated terrain (a fractal terrain) representing a mountainous terrain, and a valley. Figures \ref{fig:trajectory_helens_before} and \ref{fig:trajectory_helens_after} is heightmaps representing Mt. St. Helens before and after the erruption; this is terrains converted from actual digital elevation maps (see \cite{helens_dem} for the source DEMs). We see that generally, the trajectories seem to follow the terrain contour lines, as this minimizes the slope cost of the path, which is what was expected. 

\begin{figure}[H]
\centering
\subfloat[Autogenerated mountainous terrain]{\label{fig:trajectory_mountains}\includegraphics[width=1\textwidth]{figure/trajectory_mountains}}\\
\subfloat[Mount St. Helens (before erruption)] {\label{fig:trajectory_helens_before}\includegraphics[width=0.485\textwidth]{figure/trajectory_helens_before}}\quad
\subfloat[Mount St. Helens (after erruption)]{\label{fig:trajectory_helens_after}\includegraphics[width=0.485\textwidth]{figure/trajectory_helens_after}}
\caption{Procedurally generated road trajectories for different heightmaps}
\label{fig:road_trajectory}
\end{figure}

After loading the model into the snow simulator and adjusting the terrain, we can see the roads as it curves through the terrain. Figure \ref{fig:road_in_terrain_nosnow} shows the road and the terrain without snow cover for the random generated fractal terrain. We clearly see the effects of the terrain adjustments in the bottom two screenshots, i.e. figures \ref{fig:road_in_terrain_nosnow_2} and \ref{fig:road_in_terrain_nosnow_3}. Note that snow rendering was disabled on the overview image (figure \ref{fig:road_in_terrain_nosnow_1}) in order to improve visibility of the road. 

\begin{figure}[ht]
\centering
\subfloat[Terrain overview]{\label{fig:road_in_terrain_nosnow_1}\includegraphics[width=\textwidth]{figure/screenshots/resampled/road_nosnow_1}}\\
\subfloat[Detailed view of road]{\label{fig:road_in_terrain_nosnow_2}\includegraphics[width=0.485\textwidth]{figure/screenshots/resampled/road_nosnow_2}}\quad
\subfloat[Significant terrain adjustments]{\label{fig:road_in_terrain_nosnow_3}\includegraphics[width=0.485\textwidth]{figure/screenshots/resampled/road_nosnow_3}}
\caption{Road and terrain before snowcover}
\label{fig:road_in_terrain_nosnow}
\end{figure}

After some snow has fallen, as in figure \ref{fig:road_in_terrain_snow}, the terrain has been more or less covered in snow except for some shadowed areas. The snow height is continously "smoothed", so that snow that falls in areas with a steep gradient in the terrain, is moved downwards along the gradient, simulating sliding snow (think of it as a mini avalanche). We clearly see this effect on the road in figures \ref{fig:road_in_terrain_snow_2} and \ref{fig:road_in_terrain_snow_3}.

\begin{figure}[ht]
\centering
\subfloat[Snow covered terrain overview]{\label{fig:road_in_terrain_snow_1}\includegraphics[width=\textwidth]{figure/screenshots/resampled/road_snow_1}}\\
\subfloat[Snow from higher up covering road]{\label{fig:road_in_terrain_snow_2}\includegraphics[width=0.485\textwidth]{figure/screenshots/resampled/road_snow_2}}\quad
\subfloat[Another case of snow sliding down on the road]{\label{fig:road_in_terrain_snow_3}\includegraphics[width=0.485\textwidth]{figure/screenshots/resampled/road_snow_3}}
\caption{Road and terrain with snowcover}
\label{fig:road_in_terrain_snow}
\end{figure}

Another map that was used was Mt. St. Helens before and after the erruption. These are shown in figures \ref{fig:helens_road_closeup} and \ref{fig:helens_overview_snow}.

\begin{figure}[H]
\centering
\subfloat[Before erruption]{\label{fig:helens_road_closeup_before}\includegraphics[width=\textwidth]{figure/screenshots/resampled/helens_before_nosnow}}\\
\subfloat[After erruption]{\label{fig:helens_road_closeup_after}\includegraphics[width=\textwidth]{figure/screenshots/resampled/helens_after_nosnow}}\\
\caption{Closeup of road in the Mt. St. Helens terrain}
\label{fig:helens_road_closeup}
\end{figure}

\begin{figure}[H]
\centering
\subfloat[Before erruption]{\label{fig:helens_overview_before}\includegraphics[width=\textwidth]{figure/screenshots/resampled/helens_before_snow}}\\
\subfloat[After erruption]{\label{fig:helens_overview_after}\includegraphics[width=\textwidth]{figure/screenshots/resampled/helens_after_snow}}\\
\caption{Overlooking of Mt. St. Helens with heavy snow}
\label{fig:helens_overview_snow}
\end{figure}
\section{Performance of road generator}
In this section, I present results from performance testing of the road generator. 

\subsection{Testing parameters}
Four different height maps have been tested: digital elevation data from Mt. St. Helens before and after the erruption (from \cite{helens_dem}), a random fractal terrain representing a valley surrounded by mountains, and Trondheim. These maps range from very small to very large, with different resolutions. Table \ref{tab:testmaps} gives an overview of these maps.

\begin{table}[ht]
\centering
\begin{tabular}{llll}
\hline
\tbf{Map} & \tbf{Shorthand} & \tbf{Dimensions} & \tbf{Resolution (m)}\\
\hline
Mt. St. Helens before erruption & helens\_before & $256\times 256$ & $30$\\ 
Mt. St. Helens after erruption  & helens\_after  & $768\times 768$ & $10$\\
Random fractal terrain          & mountains      & $1024\times 1024$ & $10$\\
Trondheim                       & trondheim      & $4096\times 4096$ & $20$\\
\hline
\end{tabular}
\caption{Height maps used for performance testing}
\label{tab:testmaps}
\end{table}

The weights for road length, slope and curvature is set to 1, 100 and 100, respectively. A threshold of maximum curvature was set to 0.03, which is roughly the curvature of a circle with a radius of 33 meters. Paths with an estimated curvature of more than 0.03 is not considered, because these would give unrealistically sharp turns. A neighborhood of $k=5$ is used for all test cases.  This gives 30 neighbors per node, except for the edge nodes which have less.

\subsection{Performance of Dijkstra vs A*}
The main difference between Dijkstra's algorithm and the A* algorithm is that A* takes an estimated guess of the distance to the goal, and uses this, together with the cost to reach the next node from the start, in deciding which node to expand next. With a good heuristic, A* may expand far fewer nodes than Dijkstra's algorithm, typically resulting in a lower search time. 

In this section I present a comparison of the performance using Dijkstra's algorithm, and the A* algorithm with the heuristic in equation \ref{eq:astar_heuristic} in section \ref{sec:impl_astar}. Since the algorithms are identical except for the heuristic, Dijkstra's algorithm is simply A* with $h=0$. In the Dijkstra runs, the heuristic is therefore simply set to zero, but the algorithm is identical otherwise.

\begin{figure}[ht]
\centering
\includegraphics[width=\textwidth]{figure/plots/astar_vs_dijkstra}
\caption{A* vs. Dijkstra's algorithm. Runningtimes are relative to that of Dijkstra's algorithm}
\label{fig:astar_vs_dijkstra}
\end{figure}

\begin{table}[ht]
\centering
\begin{tabular}{|l|ll|ll|l|}
\hline
          & \multicolumn{2}{c|}{\tbf{A*}}         & \multicolumn{2}{c|}{\tbf{Dijkstra}}    & \\
\tbf{Map} & \tbf{Runningtime} & \tbf{Cost} & \tbf{Runnningtime} & \tbf{Cost} & \tbf{Speedup}\\
\hline
\input{data/astar_vs_dijkstra}
\hline
\end{tabular}
\caption{A* vs. Dijkstra's algorithm}
\label{tab:astar_vs_dijkstra}
\end{table}

The starting points for all was chosen so that a solution to the shortest path was non-trivial; a trivial setup is when the actual cost to the goal is close to the heuristic estimate of the cost to the goal; in this case, the comparison is unfair and unrealistic, as A* will almost immediately find the goal node by expanding only those nodes that the shortest path contains.

Figure \ref{fig:astar_vs_dijkstra} shows the running time of Dijkstra's algorithm relative to A*, with the runningtime of Dijkstra's algorithm defined to be one. Table \ref{tab:astar_vs_dijkstra} shows the results in tabulated form. The costs of the road generated by Dijkstra and A* is included as a confirmation that the heuristic used indeed results in an optimal trajectory.

We see that for Mt. st. Helens, there is more than two times speedup for using A*, while the Trondheim and fractal maps show more modest improvements. The Mt. St. Helens maps are generally simpler maps with only one major elevation peak, which makes the heuristic more accurate. The heuristic does not account for a rough terrain, but rather the difference in elevation between the start and end node. Both the mountains and Trondheim maps contain significant elevation changes throughout the terrain, which makes the heuristic less accurate.

\subsection{Effect of grid coarsening}
In this section I present performance results for different densities of grid points. The time used to generate the shortest path and also the optimality of the path, compared to using a density of 1, i.e. where every height point is considered a node in the search graph. The results are presented in table \ref{tab:effect_of_density}, and a plot of the relative cost increase due to grid coarsening with respect to the density is presented in figure \ref{fig:effect_of_density}. 

\begin{figure}[ht]
\centering
\includegraphics[width=\textwidth]{figure/plots/densities}
\caption{Effect of grid coarsening on optimal path cost}
\label{fig:effect_of_density}
\end{figure}

\begin{table}[ht]
\centering
\begin{tabular}{ccccc}
\hline
\tbf{Density} & \tbf{Runningtime} & \tbf{Speedup} & \tbf{Road cost} & \tbf{Difference}\\
              &                   &               &                 & \tbf{from "optimal"}\\
\hline
\multicolumn{5}{c}{Mt. St. Helens (before erruption)}\\
\hline
\input{data/helens_before}
\hline
\multicolumn{5}{c}{Mt. St. Helens (after erruption)}\\
\hline
\input{data/helens_after}
\hline
\multicolumn{5}{c}{Mountains}\\
\hline
\input{data/mountains}
\hline
\multicolumn{5}{c}{Trondheim}\\
\hline
\input{data/trondheim}
\hline
\end{tabular}
\caption{Effect of density of grid}
\label{tab:effect_of_density}
\end{table}

We see here that even though a density of one is optimal with regards to cost, which is not surprising because this gives the highest resolution in finding a path, the running time increase for using such a fine resolution is rather large. 

It is interesting to note that the real-worldmaps take a higher cost-penalty when the resolution is decreased, while the auto-generated fractal terrain (Mountains) is at most 1.22 times worse with a grid density of one node each $64\times 64$ height points than with one node for every height point. 

Fractal terrains typically have a low frequency on large elevation changes due to the way they are usually generated. Start with a flat terrain, and until some recursive depth $d$ is reached, recursively subdivide the terrain, and perturb the elevation of each subdivision; the amplitude of the perturbation typically decreases with recursion depth, giving high-frequency changes of small amplitude, and low-frequency changes of big amplitude. This means that using lower resolution on a fractal map does not neccesarily give a huge increase in cost, because the costly elevation changes usually have a low frequency, so it can be dealt with at a lower resolution.

Real world terrains also have fractal-like geometry\cite{fractalnature}, but is not procedurally generated so these terrains may (and often will) exhibit (relatively) high-frequency elevation changes with a large amplitude. This of course gives a higher penalty for using a lower resolutionas these big amplitude elevation differences may not be curcumvented by building the road around them.
