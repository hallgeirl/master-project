\chapter{A* Road Generator reference}

\section*{Synopsis}
\texttt{./pathfind [OPTIONS] <heightmap filename> <output filename w/o extension>}

\section*{Description}
pathfind is a program for creating a road trajectory through a terrain using an A* search. The output is a RoadXML file, as well as a simple list of control points. The terrain is in form of a height map read from a RAW file. The time spent finding the path as well as the total cost of the path is output to standard output right before the program exits. On completion, pathfind outputs the following files, assuming the output filename was set to "out".
\begin{description}
\item[out.rnd] \hfill \\
The RoadXML file describing the trajectory.
\item[out.rd] \hfill \\
The control points of the road trajectory in an easily parsed format.
\end{description}



\section*{Options}
The various options that can be given to pathfind is listed below. The options \texttt{-x} and \texttt{-y} are required.

\begin{description}
\item[\texttt{-x N}] \hfill \\
\textit{REQUIRED}. Width of the height map in the number of points. 

\item[\texttt{-y N}] \hfill \\
\textit{REQUIRED}. Height of the height map in the number of points. 

\item[\texttt{-a N}] \hfill \\
Height (in meters) represented by the value 0 (zero) in the height map. Default is 0.

\item[\texttt{-b N}] \hfill \\
Height (in meters) represented by the value 65535 ($2^{16}-1$) in the height map. Default is 2700.

\item[\texttt{-s N}] \hfill \\
Resolution of height map, in meters between height points. For instance, if this value is 10, then there are 10 meters between two neighboring values in the height map.

\item[\texttt{-d N}] \hfill \\
Grid coarsening factor. Setting this to anything larger than 1, will in practice give a lower resolution height map, where only one point every N will be used as nodes in the search.

\item[\texttt{-h N}] \hfill \\
Weight of the heuristic used in A*. Setting this to 0 gives Dijkstra's algorithm. Setting it to 1 gives regular A*. Setting it to a value > 1 may result in a path faster, but it might be sub-optimal with regards to cost.

\item[\texttt{-{-}startx=N}] \hfill \\
N specifies the starting X position for the search. This is relative to the top-left corner of the height map, and is in number of height points from the left edge.

\item[\texttt{-{-}starty=N}] \hfill \\
N specifies the starting Y position for the search. 

\item[\texttt{-{-}endx=N}] \hfill \\
N specifies the ending X position for the search. 

\item[\texttt{-{-}endy=N}] \hfill \\
N specifies the ending Y position for the search. 

\end{description}

\section*{Examples}

\begin{description}
\item[\texttt{./pathfind -x 768 -y 768 -a 0 -b 1500 -s 10 test.raw output}] \hfill\\
reads test.raw as the height map, maps each of the height values to a value between 0 and 1500, and outputs output.rd and output.rnd. A resolution of 10 meters is assumed. The full height map resolution is used. The default starting and ending coordinates is used, i.e. start in the top-left corner and end in bottom-right corner.

\end{description}

